\documentclass[]{article}
\usepackage{lmodern}
\usepackage{amssymb,amsmath}
\usepackage{ifxetex,ifluatex}
\usepackage{fixltx2e} % provides \textsubscript
\ifnum 0\ifxetex 1\fi\ifluatex 1\fi=0 % if pdftex
  \usepackage[T1]{fontenc}
  \usepackage[utf8]{inputenc}
\else % if luatex or xelatex
  \ifxetex
    \usepackage{mathspec}
    \usepackage{xltxtra,xunicode}
  \else
    \usepackage{fontspec}
  \fi
  \defaultfontfeatures{Mapping=tex-text,Scale=MatchLowercase}
  \newcommand{\euro}{€}
\fi
% use upquote if available, for straight quotes in verbatim environments
\IfFileExists{upquote.sty}{\usepackage{upquote}}{}
% use microtype if available
\IfFileExists{microtype.sty}{\usepackage{microtype}}{}
\usepackage[margin=1in]{geometry}
\usepackage{color}
\usepackage{fancyvrb}
\newcommand{\VerbBar}{|}
\newcommand{\VERB}{\Verb[commandchars=\\\{\}]}
\DefineVerbatimEnvironment{Highlighting}{Verbatim}{commandchars=\\\{\}}
% Add ',fontsize=\small' for more characters per line
\newenvironment{Shaded}{}{}
\newcommand{\AlertTok}[1]{\textcolor[rgb]{1.00,0.00,0.00}{\textbf{#1}}}
\newcommand{\AnnotationTok}[1]{\textcolor[rgb]{0.38,0.63,0.69}{\textbf{\textit{#1}}}}
\newcommand{\AttributeTok}[1]{\textcolor[rgb]{0.49,0.56,0.16}{#1}}
\newcommand{\BaseNTok}[1]{\textcolor[rgb]{0.25,0.63,0.44}{#1}}
\newcommand{\BuiltInTok}[1]{\textcolor[rgb]{0.00,0.50,0.00}{#1}}
\newcommand{\CharTok}[1]{\textcolor[rgb]{0.25,0.44,0.63}{#1}}
\newcommand{\CommentTok}[1]{\textcolor[rgb]{0.38,0.63,0.69}{\textit{#1}}}
\newcommand{\CommentVarTok}[1]{\textcolor[rgb]{0.38,0.63,0.69}{\textbf{\textit{#1}}}}
\newcommand{\ConstantTok}[1]{\textcolor[rgb]{0.53,0.00,0.00}{#1}}
\newcommand{\ControlFlowTok}[1]{\textcolor[rgb]{0.00,0.44,0.13}{\textbf{#1}}}
\newcommand{\DataTypeTok}[1]{\textcolor[rgb]{0.56,0.13,0.00}{#1}}
\newcommand{\DecValTok}[1]{\textcolor[rgb]{0.25,0.63,0.44}{#1}}
\newcommand{\DocumentationTok}[1]{\textcolor[rgb]{0.73,0.13,0.13}{\textit{#1}}}
\newcommand{\ErrorTok}[1]{\textcolor[rgb]{1.00,0.00,0.00}{\textbf{#1}}}
\newcommand{\ExtensionTok}[1]{#1}
\newcommand{\FloatTok}[1]{\textcolor[rgb]{0.25,0.63,0.44}{#1}}
\newcommand{\FunctionTok}[1]{\textcolor[rgb]{0.02,0.16,0.49}{#1}}
\newcommand{\ImportTok}[1]{\textcolor[rgb]{0.00,0.50,0.00}{\textbf{#1}}}
\newcommand{\InformationTok}[1]{\textcolor[rgb]{0.38,0.63,0.69}{\textbf{\textit{#1}}}}
\newcommand{\KeywordTok}[1]{\textcolor[rgb]{0.00,0.44,0.13}{\textbf{#1}}}
\newcommand{\NormalTok}[1]{#1}
\newcommand{\OperatorTok}[1]{\textcolor[rgb]{0.40,0.40,0.40}{#1}}
\newcommand{\OtherTok}[1]{\textcolor[rgb]{0.00,0.44,0.13}{#1}}
\newcommand{\PreprocessorTok}[1]{\textcolor[rgb]{0.74,0.48,0.00}{#1}}
\newcommand{\RegionMarkerTok}[1]{#1}
\newcommand{\SpecialCharTok}[1]{\textcolor[rgb]{0.25,0.44,0.63}{#1}}
\newcommand{\SpecialStringTok}[1]{\textcolor[rgb]{0.73,0.40,0.53}{#1}}
\newcommand{\StringTok}[1]{\textcolor[rgb]{0.25,0.44,0.63}{#1}}
\newcommand{\VariableTok}[1]{\textcolor[rgb]{0.10,0.09,0.49}{#1}}
\newcommand{\VerbatimStringTok}[1]{\textcolor[rgb]{0.25,0.44,0.63}{#1}}
\newcommand{\WarningTok}[1]{\textcolor[rgb]{0.38,0.63,0.69}{\textbf{\textit{#1}}}}
\ifxetex
  \usepackage[setpagesize=false, % page size defined by xetex
              unicode=false, % unicode breaks when used with xetex
              xetex]{hyperref}
\else
  \usepackage[unicode=true]{hyperref}
\fi
\hypersetup{breaklinks=true,
            bookmarks=true,
            pdfauthor={},
            pdftitle={Introducing the ``Prompt'' library},
            colorlinks=true,
            citecolor=blue,
            urlcolor=blue,
            linkcolor=magenta,
            pdfborder={0 0 0}}
\urlstyle{same}  % don't use monospace font for urls
% Make links footnotes instead of hotlinks:
\renewcommand{\href}[2]{#2\footnote{\url{#1}}}
\setlength{\parindent}{0pt}
\setlength{\parskip}{6pt plus 2pt minus 1pt}
\setlength{\emergencystretch}{3em}  % prevent overfull lines
\setcounter{secnumdepth}{0}

\title{Introducing the ``Prompt'' library}

\begin{document}
\maketitle

\% Justin Le \% June 30, 2015

\emph{Originally posted on
\textbf{\href{https://blog.jle.im/entry/introducing-the-prompt-library.html}{in
Code}}.}

\textbf{Prompt}:
\href{https://github.com/mstksg/prompt/blob/master/README.md}{README} /
\href{http://hackage.haskell.org/package/prompt}{hackage} /
\href{https://github.com/mstksg/prompt}{github}

Have you ever wanted to specify a computation involving some limited form of IO
--- like querying a database, or asking stdio --- but didn't want a computation
in the \texttt{IO} monad, opening the entire can of worms that is arbitrary
\texttt{IO}? Have you ever looked at complicated \texttt{IO\ a} you wrote last
week at 4am and prayed that it didn't launch missiles if you decided to execute
it? Do you want to be able to run an effectful computation and explicitly
\emph{say} what IO it can or cannot do?

Introducing the \emph{\href{http://hackage.haskell.org/package/prompt}{prompt}}
library! It's a small little lightweight library that allows you to specify and
describe computations involving forms of effects where you ``ask'' with a value
and receive a value in return (such as a database query, etc.), but not ever
care about how the effects are fulfilled --- freeing you from working directly
with IO.

\begin{Shaded}
\begin{Highlighting}[]
\KeywordTok{data} \DataTypeTok{Foo} \OtherTok{=} \DataTypeTok{Foo}\NormalTok{ \{}\OtherTok{ fooBar ::} \DataTypeTok{String}
\NormalTok{               ,}\OtherTok{ fooBaz ::} \DataTypeTok{Int}
\NormalTok{               \} }\KeywordTok{deriving} \DataTypeTok{Show}

\CommentTok{{-}{-} ask with a String, receive a String as an answer}
\OtherTok{promptFoo ::} \DataTypeTok{Prompt} \DataTypeTok{String} \DataTypeTok{String} \DataTypeTok{Foo}
\NormalTok{promptFoo }\OtherTok{=} \DataTypeTok{Foo}
        \OperatorTok{\textless{}$\textgreater{}}\NormalTok{ prompt }\StringTok{"bar"}
        \OperatorTok{\textless{}*\textgreater{}} \FunctionTok{fmap} \FunctionTok{length}\NormalTok{ (prompt }\StringTok{"baz"}\NormalTok{)}
\end{Highlighting}
\end{Shaded}

\section{Running}\label{running}

You can now ``run it'' in IO, by talking to stdio ---

\begin{Shaded}
\begin{Highlighting}[]
\NormalTok{ghci}\OperatorTok{\textgreater{}}\NormalTok{ runPromptM promptFoo }\OperatorTok{$}\NormalTok{ \textbackslash{}str }\OtherTok{{-}\textgreater{}} \FunctionTok{putStrLn}\NormalTok{ str }\OperatorTok{\textgreater{}\textgreater{}} \FunctionTok{getLine}
\NormalTok{bar                 }\CommentTok{{-}{-} stdout prompt}
\OperatorTok{\textgreater{}}\NormalTok{ hello}\OperatorTok{!}            \CommentTok{{-}{-} stdin response typed in}
\NormalTok{baz                 }\CommentTok{{-}{-} stdout prompt}
\OperatorTok{\textgreater{}}\NormalTok{ i am baz          }\CommentTok{{-}{-} stdin response typed in}
\DataTypeTok{Foo} \StringTok{"hello!"} \DecValTok{8}      \CommentTok{{-}{-} result}
\end{Highlighting}
\end{Shaded}

(this is also just \texttt{interactP\ promptFoo})

Or you can maybe request it from the environment variables:

\begin{Shaded}
\begin{Highlighting}[]
\NormalTok{ghci}\OperatorTok{\textgreater{}} \KeywordTok{import} \DataTypeTok{System.Environment}
\NormalTok{ghci}\OperatorTok{\textgreater{}}\NormalTok{ setEnv }\StringTok{"bar"} \StringTok{"hello!"}
\NormalTok{ghci}\OperatorTok{\textgreater{}}\NormalTok{ setEnv }\StringTok{"baz"} \StringTok{"i am baz"}
\NormalTok{ghci}\OperatorTok{\textgreater{}}\NormalTok{ runPromptM promptFoo getEnv}
\DataTypeTok{Foo} \StringTok{"hello!"} \DecValTok{8}
\end{Highlighting}
\end{Shaded}

Or maybe you want to fulfill the prompts purely:

\begin{Shaded}
\begin{Highlighting}[]
\NormalTok{ghci}\OperatorTok{\textgreater{}} \KeywordTok{import} \KeywordTok{qualified} \DataTypeTok{Data.Map} \KeywordTok{as} \DataTypeTok{M}
\NormalTok{ghci}\OperatorTok{\textgreater{}} \KeywordTok{let}\NormalTok{ testMap }\OtherTok{=}\NormalTok{ M.fromList [(}\StringTok{"bar"}\NormalTok{, }\StringTok{"hello!"}\NormalTok{), (}\StringTok{"baz"}\NormalTok{, }\StringTok{"i am baz"}\NormalTok{)]}
\NormalTok{ghci}\OperatorTok{\textgreater{}}\NormalTok{ runPrompt promptFoo (testMap }\OperatorTok{M.!}\NormalTok{)}
\DataTypeTok{Foo} \StringTok{"hello!"} \DecValTok{8}
\end{Highlighting}
\end{Shaded}

With \texttt{Prompt}, specify the computation and your logic \emph{without
involving any IO}, so you can write safe code without arbitrary side effects. If
you ever receive a \texttt{Prompt}, you know it can't wipe out your hard drive
or do any IO other than exactly what you allow it to do! I'd feel more safe
running a \texttt{Prompt\ a\ b\ r} than an \texttt{IO\ r}.

You can also do some cute tricks; \texttt{Prompt\ a\ ()\ r} with a ``prompt
response function'' like \texttt{putStrLn} lets you do streaming logging, and
defer \emph{how} the logging is done --- to IO, to a list?

\begin{Shaded}
\begin{Highlighting}[]
\NormalTok{ghci}\OperatorTok{\textgreater{}} \KeywordTok{let}\NormalTok{ logHelloWord }\OtherTok{=} \FunctionTok{mapM\_}\NormalTok{ prompt [}\StringTok{"hello"}\NormalTok{, }\StringTok{"world"}\NormalTok{]}
\NormalTok{ghci}\OperatorTok{\textgreater{}}\NormalTok{ runPromptM logHelloWorld }\FunctionTok{putStrLn}
\NormalTok{hello}
\NormalTok{world}
\NormalTok{ghci}\OperatorTok{\textgreater{}}\NormalTok{ execWriter }\OperatorTok{$}\NormalTok{ runPromptM logHelloWorld tell}
\StringTok{"helloworld"}
\end{Highlighting}
\end{Shaded}

\texttt{Prompt\ ()\ b\ r} is like a fancy \texttt{ReaderT\ b\ m\ r}, where you
``defer'' the choice of the Monad.

\section{Combining with other effects}\label{combining-with-other-effects}

\texttt{Prompt} can be used as an underlying ``effects'' source for libraries
like \emph{pipes}, \emph{conduit}, and \emph{auto}. If your effects are only
ever asking and prompting and receiving, there's really no need to put the
entire power of \texttt{IO} underneath your DSL as an effects source. That's
just crazy!

\texttt{Prompt} can be used with monad transformers to give you safe underlying
effect sources, like \texttt{StateT\ s\ (Prompt\ a\ b)\ r}, which is a stateful
computation which can sometimes sequence ``prompty'' effects. \texttt{Prompt} is
also itself a ``Traversable transformer'', with \texttt{PrompT\ a\ b\ t\ r}. It
can perform computations in the context of a Traversable \texttt{t}, to be able
to incorporate built-in short-circuiting and logging, etc.

This is all abstracted over with \texttt{MonadPrompt}, \texttt{MonadError},
\texttt{MonadPlus}, etc., typeclasses ---

\begin{Shaded}
\begin{Highlighting}[]
\OtherTok{promptFoo2 ::}\NormalTok{ (}\DataTypeTok{MonadPlus}\NormalTok{ m, }\DataTypeTok{MonadPrompt} \DataTypeTok{String} \DataTypeTok{String}\NormalTok{ m) }\OtherTok{=\textgreater{}}\NormalTok{ m }\DataTypeTok{Foo}
\NormalTok{promptFoo2 }\OtherTok{=} \KeywordTok{do}
\NormalTok{    bar }\OtherTok{\textless{}{-}}\NormalTok{ prompt }\StringTok{"bar"}
\NormalTok{    str }\OtherTok{\textless{}{-}}\NormalTok{ prompt }\StringTok{"baz"}
    \KeywordTok{case}\NormalTok{ readMaybe str }\KeywordTok{of}
        \DataTypeTok{Just}\NormalTok{ baz }\OtherTok{{-}\textgreater{}} \FunctionTok{return} \OperatorTok{$} \DataTypeTok{Foo}\NormalTok{ bar baz}
        \DataTypeTok{Nothing}  \OtherTok{{-}\textgreater{}}\NormalTok{ mzero}

\CommentTok{{-}{-} more polymorphic}
\OtherTok{promptFoo ::} \DataTypeTok{MonadPrompt} \DataTypeTok{String} \DataTypeTok{String}\NormalTok{ m }\OtherTok{=\textgreater{}}\NormalTok{ m }\DataTypeTok{Foo}
\NormalTok{promptFoo }\OtherTok{=} \DataTypeTok{Foo}
        \OperatorTok{\textless{}$\textgreater{}}\NormalTok{ prompt }\StringTok{"bar"}
        \OperatorTok{\textless{}*\textgreater{}} \FunctionTok{fmap} \FunctionTok{length}\NormalTok{ (prompt }\StringTok{"baz"}\NormalTok{)}
\end{Highlighting}
\end{Shaded}

You can run \texttt{promptFoo} as a
\texttt{MaybeT\ (Prompt\ String\ String)\ Foo}, and manually unwrap:

\begin{Shaded}
\begin{Highlighting}[]
\NormalTok{ghci}\OperatorTok{\textgreater{}}\NormalTok{ interactP }\OperatorTok{.}\NormalTok{ runMaybeT }\OperatorTok{$}\NormalTok{ promptFoo2}
\NormalTok{bar}
\OperatorTok{\textgreater{}}\NormalTok{ hello}\OperatorTok{!}
\NormalTok{baz}
\OperatorTok{\textgreater{}}\NormalTok{ i am baz}
\DataTypeTok{Nothing}
\NormalTok{ghci}\OperatorTok{\textgreater{}}\NormalTok{ interactP }\OperatorTok{.}\NormalTok{ runMaybeT }\OperatorTok{$}\NormalTok{ promptFoo2}
\NormalTok{bar}
\OperatorTok{\textgreater{}}\NormalTok{ hello}\OperatorTok{!}
\NormalTok{baz}
\OperatorTok{\textgreater{}} \DecValTok{19}
\DataTypeTok{Just}\NormalTok{ (}\DataTypeTok{Foo} \StringTok{"hello!"} \DecValTok{19}\NormalTok{)}
\end{Highlighting}
\end{Shaded}

Or you can run it as a \texttt{PromptT\ String\ String\ MaybeT\ Foo}, to have
\texttt{PromptT} handle the wrapping/unwrapping itself:

\begin{Shaded}
\begin{Highlighting}[]
\NormalTok{ghci}\OperatorTok{\textgreater{}}\NormalTok{ interactPT promptFoo2}
\NormalTok{bar}
\OperatorTok{\textgreater{}}\NormalTok{ hello}\OperatorTok{!}
\NormalTok{baz}
\OperatorTok{\textgreater{}}\NormalTok{ i am baz}
\DataTypeTok{Nothing}
\NormalTok{ghci}\OperatorTok{\textgreater{}}\NormalTok{ interactPT }\OperatorTok{$}\NormalTok{ promptFoo2 }\OperatorTok{\textless{}|\textgreater{}}\NormalTok{ promptFoo}
\NormalTok{bar}
\OperatorTok{\textgreater{}}\NormalTok{ hello}\OperatorTok{!}
\NormalTok{baz}
\OperatorTok{\textgreater{}}\NormalTok{ i am baz}
\NormalTok{bar                 }\CommentTok{{-}{-} failed to parse {-}{-}{-} retrying with promptFoo!}
\OperatorTok{\textgreater{}}\NormalTok{ hello}\OperatorTok{!}
\NormalTok{baz}
\OperatorTok{\textgreater{}}\NormalTok{ i am baz}
\DataTypeTok{Just}\NormalTok{ (}\DataTypeTok{Foo} \StringTok{"hello"} \DecValTok{8}\NormalTok{)}
\end{Highlighting}
\end{Shaded}

The previous example of \texttt{logHelloWorld}?

\begin{Shaded}
\begin{Highlighting}[]
\NormalTok{ghci}\OperatorTok{\textgreater{}}\NormalTok{ runPromptT (}\OtherTok{logHelloWorld ::} \DataTypeTok{PromptT} \DataTypeTok{String}\NormalTok{ () (}\DataTypeTok{Writer} \DataTypeTok{String}\NormalTok{) ()) tell}
\StringTok{"helloworld"}
\end{Highlighting}
\end{Shaded}

\section{Runners}\label{runners}

The ``runners'' are:

\begin{Shaded}
\begin{Highlighting}[]
\OtherTok{interactP   ::}                  \DataTypeTok{Prompt}  \DataTypeTok{String} \DataTypeTok{String}\NormalTok{   r }\OtherTok{{-}\textgreater{}} \DataTypeTok{IO}\NormalTok{ r}
\OtherTok{interactPT  ::} \DataTypeTok{Applicative}\NormalTok{ t }\OtherTok{=\textgreater{}} \DataTypeTok{PromptT} \DataTypeTok{String} \DataTypeTok{String}\NormalTok{ t r }\OtherTok{{-}\textgreater{}} \DataTypeTok{IO}\NormalTok{ (t r)}

\OtherTok{runPrompt   ::}                  \DataTypeTok{Prompt}\NormalTok{  a b   r }\OtherTok{{-}\textgreater{}}\NormalTok{ (a }\OtherTok{{-}\textgreater{}}\NormalTok{   b) }\OtherTok{{-}\textgreater{}}\NormalTok{ r}
\OtherTok{runPromptM  ::} \DataTypeTok{Monad}\NormalTok{ m       }\OtherTok{=\textgreater{}} \DataTypeTok{Prompt}\NormalTok{  a b   r }\OtherTok{{-}\textgreater{}}\NormalTok{ (a }\OtherTok{{-}\textgreater{}}\NormalTok{ m b) }\OtherTok{{-}\textgreater{}}\NormalTok{ m r}

\OtherTok{runPromptT  ::}                  \DataTypeTok{PromptT}\NormalTok{ a b t r }\OtherTok{{-}\textgreater{}}\NormalTok{ (a }\OtherTok{{-}\textgreater{}}\NormalTok{    t b)  }\OtherTok{{-}\textgreater{}}\NormalTok{ t r}
\OtherTok{runPromptTM ::} \DataTypeTok{Monad}\NormalTok{ m       }\OtherTok{=\textgreater{}} \DataTypeTok{PromptT}\NormalTok{ a b t r }\OtherTok{{-}\textgreater{}}\NormalTok{ (a }\OtherTok{{-}\textgreater{}}\NormalTok{ m (t b)) }\OtherTok{{-}\textgreater{}}\NormalTok{ m (t r)}
\end{Highlighting}
\end{Shaded}

Note that \texttt{runPromptM} and \texttt{runPromptTM} can run in monads (like
\texttt{IO}) that are \emph{completely unrelated} to the \texttt{Prompt} type
itself. It sequences them all ``after the fact''. It's also interesting to note
that \texttt{runPrompt} is just a glorified
\texttt{Reader\ (a\ -\textgreater{}\ b)\ r}.

With \texttt{runPromptTM}, you can incorporate \texttt{t} in your ``prompt
response'' function, too. Which brings us to our grand finale -- environment
variable parsing!

\begin{Shaded}
\begin{Highlighting}[]
\KeywordTok{import} \DataTypeTok{Control.Monad.Error.Class}
\KeywordTok{import} \DataTypeTok{Control.Monad.Prompt}
\KeywordTok{import} \DataTypeTok{Text.Read}
\KeywordTok{import} \KeywordTok{qualified} \DataTypeTok{Data.Map} \KeywordTok{as} \DataTypeTok{M}

\KeywordTok{type} \DataTypeTok{Key} \OtherTok{=} \DataTypeTok{String}
\KeywordTok{type} \DataTypeTok{Val} \OtherTok{=} \DataTypeTok{String}

\KeywordTok{data} \DataTypeTok{MyError} \OtherTok{=} \DataTypeTok{MENoParse} \DataTypeTok{Key} \DataTypeTok{Val}
             \OperatorTok{|} \DataTypeTok{MENotFound} \DataTypeTok{Key}
             \KeywordTok{deriving} \DataTypeTok{Show}

\OtherTok{promptRead ::}\NormalTok{ (}\DataTypeTok{MonadError} \DataTypeTok{MyError}\NormalTok{ m, }\DataTypeTok{MonadPrompt} \DataTypeTok{Key} \DataTypeTok{Val}\NormalTok{ m, }\DataTypeTok{Read}\NormalTok{ b)}
           \OtherTok{=\textgreater{}} \DataTypeTok{Key} \OtherTok{{-}\textgreater{}}\NormalTok{ m b}
\CommentTok{{-}{-} promptRead :: Read b =\textgreater{} Key {-}\textgreater{} PromptT Key Val (Either MyError) b}
\NormalTok{promptRead k }\OtherTok{=} \KeywordTok{do}
\NormalTok{    resp }\OtherTok{\textless{}{-}}\NormalTok{ prompt k}
    \KeywordTok{case}\NormalTok{ readMaybe resp }\KeywordTok{of}
      \DataTypeTok{Nothing} \OtherTok{{-}\textgreater{}}\NormalTok{ throwError }\OperatorTok{$} \DataTypeTok{MEParse}\NormalTok{ k resp}
      \DataTypeTok{Just}\NormalTok{ v  }\OtherTok{{-}\textgreater{}} \FunctionTok{return}\NormalTok{ v}

\OtherTok{promptFoo3 ::} \DataTypeTok{MonadPrompt} \DataTypeTok{Key} \DataTypeTok{Val}\NormalTok{ m }\OtherTok{=\textgreater{}}\NormalTok{ m }\DataTypeTok{Foo}
\CommentTok{{-}{-} promptFoo3 :: Applicative t =\textgreater{} PromptT Key Val t Foo}
\NormalTok{promptFoo3 }\OtherTok{=} \DataTypeTok{Foo} \OperatorTok{\textless{}$\textgreater{}}\NormalTok{ prompt }\StringTok{"bar"} \OperatorTok{\textless{}*\textgreater{}}\NormalTok{ promptRead }\StringTok{"baz"}

\CommentTok{{-}{-}}
\CommentTok{{-}{-} running!}

\CommentTok{{-}{-} Lookup environment variables, and "throw" an error if not found}
\OtherTok{throughEnv ::} \DataTypeTok{IO}\NormalTok{ (}\DataTypeTok{Either} \DataTypeTok{MyError} \DataTypeTok{Foo}\NormalTok{)}
\NormalTok{throughEnv }\OtherTok{=}\NormalTok{ runPromptTM parseFoo3 }\OperatorTok{$}\NormalTok{ \textbackslash{}k }\OtherTok{{-}\textgreater{}} \KeywordTok{do}
\NormalTok{    env }\OtherTok{\textless{}{-}}\NormalTok{ lookupEnv k}
    \FunctionTok{return} \OperatorTok{$} \KeywordTok{case}\NormalTok{ env }\KeywordTok{of}
      \DataTypeTok{Nothing} \OtherTok{{-}\textgreater{}} \DataTypeTok{Left}\NormalTok{ (}\DataTypeTok{MENotFound}\NormalTok{ k)}
      \DataTypeTok{Just}\NormalTok{ v  }\OtherTok{{-}\textgreater{}} \DataTypeTok{Right}\NormalTok{ v}

\CommentTok{{-}{-} Fulfill the prompt through user input}
\OtherTok{throughStdIO ::} \DataTypeTok{IO}\NormalTok{ (}\DataTypeTok{Either} \DataTypeTok{MyError} \DataTypeTok{Foo}\NormalTok{)}
\NormalTok{throughStdIO }\OtherTok{=}\NormalTok{ interactPT parseFoo3}

\CommentTok{{-}{-} Fulfill the prompt through user input; count blank responses as "not found"}
\OtherTok{throughStdIOBlankIsError ::} \DataTypeTok{IO}\NormalTok{ (}\DataTypeTok{Either} \DataTypeTok{MyError} \DataTypeTok{Foo}\NormalTok{)}
\NormalTok{throughStdIOBlankIsError }\OtherTok{=}\NormalTok{ runPromptTM parseFoo3 }\OperatorTok{$}\NormalTok{ \textbackslash{}k }\OtherTok{{-}\textgreater{}} \KeywordTok{do}
    \FunctionTok{putStrLn}\NormalTok{ k}
\NormalTok{    resp }\OtherTok{\textless{}{-}} \FunctionTok{getLine}
    \FunctionTok{return} \OperatorTok{$} \KeywordTok{if} \FunctionTok{null}\NormalTok{ resp}
      \KeywordTok{then} \DataTypeTok{Left}\NormalTok{ (}\DataTypeTok{MENotFound}\NormalTok{ k)}
      \KeywordTok{else} \DataTypeTok{Right}\NormalTok{ resp}

\CommentTok{{-}{-} Fulfill the prompt purely through a Map lookup}
\OtherTok{throughMap ::} \DataTypeTok{M.Map} \DataTypeTok{Key} \DataTypeTok{Val} \OtherTok{{-}\textgreater{}} \DataTypeTok{Either} \DataTypeTok{MyError} \DataTypeTok{Foo}
\NormalTok{throughMap m }\OtherTok{=}\NormalTok{ runPromptT parseFoo3 }\OperatorTok{$}\NormalTok{ \textbackslash{}k }\OtherTok{{-}\textgreater{}}
    \KeywordTok{case}\NormalTok{ M.lookup k m }\KeywordTok{of}
      \DataTypeTok{Nothing} \OtherTok{{-}\textgreater{}} \DataTypeTok{Left}\NormalTok{ (}\DataTypeTok{MENotFound}\NormalTok{ k)}
      \DataTypeTok{Just}\NormalTok{ v  }\OtherTok{{-}\textgreater{}} \DataTypeTok{Right}\NormalTok{ v}
\end{Highlighting}
\end{Shaded}

Hope you enjoy! Please feel free to leave a comment, find me on
\href{https://twitter.com/mstk}{twitter}, leave an issue on the
\href{https://github.com/mstksg/prompt}{github}, etc. --- and I'm usually on
freenode's \emph{\#haskell} as \emph{jle`} if you have any questions!

\section{Comparisons}\label{comparisons}

To lay it all on the floor,

\begin{Shaded}
\begin{Highlighting}[]
\KeywordTok{newtype} \DataTypeTok{PromptT}\NormalTok{ a b t r }\OtherTok{=} \DataTypeTok{PromptT}\NormalTok{ \{}\OtherTok{ runPromptTM ::} \KeywordTok{forall}\NormalTok{ m}\OperatorTok{.} \DataTypeTok{Monad}\NormalTok{ m }\OtherTok{=\textgreater{}}\NormalTok{ (a }\OtherTok{{-}\textgreater{}}\NormalTok{ m (t b)) }\OtherTok{{-}\textgreater{}}\NormalTok{ m (t r) \}}
\end{Highlighting}
\end{Shaded}

There is admittedly a popular misconception that I've seen going around that
equates this sort of type to \texttt{Free} from the \emph{free} package.
However, \texttt{Free} doesn't really have anything significant to do with this.
Sure, you might be able to generate this type by using \texttt{FreeT} over a
specifically chosen Functor, but\ldots this is the case for literally any Monad
ever, so that doesn't really mean much :)

It's also unrelated in this same manner to \texttt{Prompt} from the
\emph{MonadPrompt} package, and \texttt{Program} from \emph{operational} too.

One close relative to this type is
\texttt{forall\ m.\ ReaderT\ (a\ -\textgreater{}\ m\ b)\ m\ r}, where
\texttt{prompt\ k\ =\ ReaderT\ (\$\ k)}. This is more or less equivalent to
\texttt{Prompt}, but still can't do the things that \texttt{PromptT} can do
without a special instance of Monad.

This type is also similar in structure to \texttt{Bazaar}, from the \emph{lens}
package. The biggest difference that makes \texttt{Bazaar} unusable is because
the RankN constraint is only \texttt{Applicative}, not \texttt{Monad}, so a
\texttt{Monad} instance is impossible. Ignoring that (or if it's okay for you to
only use the \texttt{Applicative} instance), \texttt{Bazaar} forces the
``prompting effect'' to take place in the same context as the
\texttt{Traversable} \texttt{t}\ldots which really defeats the purpose of this
whole thing in the first place (the idea is to be able to separate your
prompting effect from your application logic). If the \texttt{Traversable} you
want to transform has a ``monad transformer'' version, then you can somewhat
simulate \texttt{PromptT} for that specifc \texttt{t} with the transformer
version.

It's also somewhat similar to the \texttt{Client} type from \emph{pipes}, but
it's also a bit tricky to use that with a different effect type than the logic
\texttt{Traversable}, as well\ldots so it has a lot of the same difference as
\texttt{Bazaar} here.

But this type is common/simple enough that I'm sure someone has it somewhere in
a library that I haven't been able to find. If you find it, let me know!

\section{Signoff}\label{signoff}

Hi, thanks for reading! You can reach me via email at
\href{mailto:justin@jle.im}{\nolinkurl{justin@jle.im}}, or at twitter at
\href{https://twitter.com/mstk}{@mstk}! This post and all others are published
under the \href{https://creativecommons.org/licenses/by-nc-nd/3.0/}{CC-BY-NC-ND
3.0} license. Corrections and edits via pull request are welcome and encouraged
at \href{https://github.com/mstksg/inCode}{the source repository}.

If you feel inclined, or this post was particularly helpful for you, why not
consider \href{https://www.patreon.com/justinle/overview}{supporting me on
Patreon}, or a \href{bitcoin:3D7rmAYgbDnp4gp4rf22THsGt74fNucPDU}{BTC donation}?
:)

\end{document}
