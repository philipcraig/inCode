\documentclass[]{article}
\usepackage{lmodern}
\usepackage{amssymb,amsmath}
\usepackage{ifxetex,ifluatex}
\usepackage{fixltx2e} % provides \textsubscript
\ifnum 0\ifxetex 1\fi\ifluatex 1\fi=0 % if pdftex
  \usepackage[T1]{fontenc}
  \usepackage[utf8]{inputenc}
\else % if luatex or xelatex
  \ifxetex
    \usepackage{mathspec}
    \usepackage{xltxtra,xunicode}
  \else
    \usepackage{fontspec}
  \fi
  \defaultfontfeatures{Mapping=tex-text,Scale=MatchLowercase}
  \newcommand{\euro}{€}
\fi
% use upquote if available, for straight quotes in verbatim environments
\IfFileExists{upquote.sty}{\usepackage{upquote}}{}
% use microtype if available
\IfFileExists{microtype.sty}{\usepackage{microtype}}{}
\usepackage[margin=1in]{geometry}
\usepackage{color}
\usepackage{fancyvrb}
\newcommand{\VerbBar}{|}
\newcommand{\VERB}{\Verb[commandchars=\\\{\}]}
\DefineVerbatimEnvironment{Highlighting}{Verbatim}{commandchars=\\\{\}}
% Add ',fontsize=\small' for more characters per line
\newenvironment{Shaded}{}{}
\newcommand{\AlertTok}[1]{\textcolor[rgb]{1.00,0.00,0.00}{\textbf{#1}}}
\newcommand{\AnnotationTok}[1]{\textcolor[rgb]{0.38,0.63,0.69}{\textbf{\textit{#1}}}}
\newcommand{\AttributeTok}[1]{\textcolor[rgb]{0.49,0.56,0.16}{#1}}
\newcommand{\BaseNTok}[1]{\textcolor[rgb]{0.25,0.63,0.44}{#1}}
\newcommand{\BuiltInTok}[1]{\textcolor[rgb]{0.00,0.50,0.00}{#1}}
\newcommand{\CharTok}[1]{\textcolor[rgb]{0.25,0.44,0.63}{#1}}
\newcommand{\CommentTok}[1]{\textcolor[rgb]{0.38,0.63,0.69}{\textit{#1}}}
\newcommand{\CommentVarTok}[1]{\textcolor[rgb]{0.38,0.63,0.69}{\textbf{\textit{#1}}}}
\newcommand{\ConstantTok}[1]{\textcolor[rgb]{0.53,0.00,0.00}{#1}}
\newcommand{\ControlFlowTok}[1]{\textcolor[rgb]{0.00,0.44,0.13}{\textbf{#1}}}
\newcommand{\DataTypeTok}[1]{\textcolor[rgb]{0.56,0.13,0.00}{#1}}
\newcommand{\DecValTok}[1]{\textcolor[rgb]{0.25,0.63,0.44}{#1}}
\newcommand{\DocumentationTok}[1]{\textcolor[rgb]{0.73,0.13,0.13}{\textit{#1}}}
\newcommand{\ErrorTok}[1]{\textcolor[rgb]{1.00,0.00,0.00}{\textbf{#1}}}
\newcommand{\ExtensionTok}[1]{#1}
\newcommand{\FloatTok}[1]{\textcolor[rgb]{0.25,0.63,0.44}{#1}}
\newcommand{\FunctionTok}[1]{\textcolor[rgb]{0.02,0.16,0.49}{#1}}
\newcommand{\ImportTok}[1]{\textcolor[rgb]{0.00,0.50,0.00}{\textbf{#1}}}
\newcommand{\InformationTok}[1]{\textcolor[rgb]{0.38,0.63,0.69}{\textbf{\textit{#1}}}}
\newcommand{\KeywordTok}[1]{\textcolor[rgb]{0.00,0.44,0.13}{\textbf{#1}}}
\newcommand{\NormalTok}[1]{#1}
\newcommand{\OperatorTok}[1]{\textcolor[rgb]{0.40,0.40,0.40}{#1}}
\newcommand{\OtherTok}[1]{\textcolor[rgb]{0.00,0.44,0.13}{#1}}
\newcommand{\PreprocessorTok}[1]{\textcolor[rgb]{0.74,0.48,0.00}{#1}}
\newcommand{\RegionMarkerTok}[1]{#1}
\newcommand{\SpecialCharTok}[1]{\textcolor[rgb]{0.25,0.44,0.63}{#1}}
\newcommand{\SpecialStringTok}[1]{\textcolor[rgb]{0.73,0.40,0.53}{#1}}
\newcommand{\StringTok}[1]{\textcolor[rgb]{0.25,0.44,0.63}{#1}}
\newcommand{\VariableTok}[1]{\textcolor[rgb]{0.10,0.09,0.49}{#1}}
\newcommand{\VerbatimStringTok}[1]{\textcolor[rgb]{0.25,0.44,0.63}{#1}}
\newcommand{\WarningTok}[1]{\textcolor[rgb]{0.38,0.63,0.69}{\textbf{\textit{#1}}}}
\ifxetex
  \usepackage[setpagesize=false, % page size defined by xetex
              unicode=false, % unicode breaks when used with xetex
              xetex]{hyperref}
\else
  \usepackage[unicode=true]{hyperref}
\fi
\hypersetup{breaklinks=true,
            bookmarks=true,
            pdfauthor={},
            pdftitle={First-Class ``Statements''},
            colorlinks=true,
            citecolor=blue,
            urlcolor=blue,
            linkcolor=magenta,
            pdfborder={0 0 0}}
\urlstyle{same}  % don't use monospace font for urls
% Make links footnotes instead of hotlinks:
\renewcommand{\href}[2]{#2\footnote{\url{#1}}}
\setlength{\parindent}{0pt}
\setlength{\parskip}{6pt plus 2pt minus 1pt}
\setlength{\emergencystretch}{3em}  % prevent overfull lines
\setcounter{secnumdepth}{0}

\title{First-Class ``Statements''}

\begin{document}
\maketitle

\% Justin Le \% July 28, 2014

\emph{Originally posted on
\textbf{\href{https://blog.jle.im/entry/first-class-statements.html}{in Code}}.}

One thing I've really always appreciated about Haskell is that all
``statements'' in Haskell (or at least, what would be statements in other
languages) are first-class members of the language. That is, (imperative)
statements are literally just normal objects (no different from numbers, or
lists, or booleans) --- they can be saved to variables, passed to functions,
transformed using normal functions, copied, etc. Haskell doesn't have statements
--- everything is an expression, representing normal data! This really opens up
a whole world of possibilities for not only reasoning about your code, but also
for new ways to frame ideas in contexts of parallelism, concurrency, exceptions,
DSLs, and more.

To clarify, by ``statement'', I mean it in the sense of a ``command'' from
traditional imperative programming that, when control flow reaches it, executes
some sort of action or modification to some state. The
\href{http://en.wikipedia.org/wiki/Statement_(computer_science)}{wikipedia
article} has a nice explanation. Some typical statements from common imperative
languages include:

\begin{Shaded}
\begin{Highlighting}[]
\DataTypeTok{int}\NormalTok{ a }\OperatorTok{=} \DecValTok{4}\OperatorTok{;}              \CommentTok{// declaration \& assignment}
\NormalTok{a }\OperatorTok{+=} \DecValTok{5}\OperatorTok{;}                 \CommentTok{// modification}
\NormalTok{printf}\OperatorTok{(}\StringTok{"hello world"}\OperatorTok{);}  \CommentTok{// call}
\ControlFlowTok{return} \KeywordTok{false}\OperatorTok{;}           \CommentTok{// exit points}
\end{Highlighting}
\end{Shaded}

In these languages, whenever control flow reaches these statements, something
happens. We do not differentiate the act of \emph{evaluating} these statements
(figuring out what they are) from \emph{executing} these statements. Something
just \emph{happens} when you see an assignment.

It is clear that in these languages, something about these statements are
magical or a special part of the language. They are wholly different than, say,
an integer, or a boolean. They aren't normal ``objects'' or ``data'' in your
system.

Even if your programming languages have first-class functions, \texttt{printf}
might be a first-class value, but the \emph{call} of it (usually indicated with
parentheses, like \texttt{printf()}) is definitely something\ldots different
altogether. You \emph{can} simulate this in languages by creating a sub-language
inside the language, but you're always going to have an interplay between the
two. There will always be the dichotomy between statements and data.

\section{Statements as data}\label{statements-as-data}

In Haskell, \texttt{putStrLn\ "hello\ world"} is literally just a normal, boring
object, of type \texttt{IO\ ()} (If you come from an OOP background,
\texttt{IO\ a} is sort of like a generic/template,
\texttt{IO\ \textless{}a\textgreater{}}). Like an \texttt{Int}, or
\texttt{Bool}, or a \texttt{String}, or a \texttt{{[}Double{]}} (a list of
\texttt{Double}). \emph{Evaluating} it doesn't really do anything. It's like
evaluating the number \texttt{1}, or the expression \texttt{2\ +\ 5}. Cool ---
you evaluated \texttt{2\ +\ 5} into a \texttt{7}; now what? Does anything
happen? Not really. It's still just an \texttt{Int}.

\texttt{putStrLn\ "hello\ world"} is just a normal data/term/value that
represents (through some abstract representation that isn't really important)
the act of a \emph{computer} printing the string \texttt{"hello\ world"} to
stdout.

\begin{center}\rule{0.5\linewidth}{0.5pt}\end{center}

\textbf{Aside}

The type \texttt{IO\ ()} means ``an object/abstracted data structure that
represents the act of a computer computing a \texttt{()}'' --- or, in other
terms, ``instructions for a computer to produce a \texttt{()}''. \texttt{()} is
sort of like the empty tuple\ldots it's a type that is only inhabited by
just\ldots well, \texttt{()}. You can think of \texttt{()} as being analogous to
a function returning ``void'' in other languages.

If you have experience with object-oriented languages and templates/generics,
\texttt{IO\ a} sort of corresponds to something like
\texttt{IO\textless{}a\textgreater{}}.

There are definitely many values of type \texttt{IO\ Int} or
\texttt{IO\ String}, which represent actions that produce an \texttt{Int} or a
\texttt{String}, respectively. One common example is \texttt{getLine}, which is
of type \texttt{IO\ String} --- \texttt{getLine} is an object that represents
the act of a computer getting input from stdin; the ``result'' of this action is
a \texttt{String}. An \texttt{IO\ Int} would represent a CPU
computation/IO-based computation that produces an \texttt{Int}.

For the sake of this discussion, we'll only be considering
\texttt{IO\ ()}s\ldots but in real life, these other types pop up just as often.

\begin{center}\rule{0.5\linewidth}{0.5pt}\end{center}

Haskell gives you a bunch of \emph{combinators}/functions to \emph{work} with
these \texttt{IO\ ()}'s (and \texttt{IO\ a}'s in general). To manipulate then,
merge them, sequence them, compose them\ldots anything you can dream of!

The most popular and common combinator is
\texttt{(\textgreater{}\textgreater{})}, which is usually used as an infix
operator. A common use case: Say you want to create an IO action that prints
``hello'', then ``world''. But you only have \texttt{putStrLn\ "hello"}, which
represents printing ``hello'', and \texttt{putStrLn\ "world"}, which represents
printing ``world''.

If you have those two \texttt{IO\ ()}s, you can use the
\texttt{(\textgreater{}\textgreater{})} combinator to ``merge'' them and create
a new \texttt{IO\ ()}.\footnote{The type of
  \texttt{(\textgreater{}\textgreater{})} is actually more general:
  \texttt{(\textgreater{}\textgreater{})\ ::\ IO\ a\ -\textgreater{}\ IO\ b\ -\textgreater{}\ IO\ b}.
  We do only use it for \texttt{IO\ ()} in this post, though.

  Actually, would you be mad at me if I told you that its true type in Haskell
  is even more general? It's actually
  \texttt{m\ a\ -\textgreater{}\ m\ b\ -\textgreater{}\ m\ b}, for any type
  \texttt{m} that is a member of the \texttt{Monad} typeclass. You can think of
  it like any type that ``implements'' \texttt{(\textgreater{}\textgreater{})}
  for that type, and more, following some rules on how it should behave.

  Ah! This rabbit hole is a bit deep though, so I recommend you not to worry
  about it!} In this case:

\begin{Shaded}
\begin{Highlighting}[]
\CommentTok{{-}{-} :: means "has the type"}

\CommentTok{{-}{-} putStrLn "hello" is an object with type IO ().}
\FunctionTok{putStrLn} \StringTok{"hello"}\OtherTok{ ::} \DataTypeTok{IO}\NormalTok{ ()}
\FunctionTok{putStrLn} \StringTok{"world"}\OtherTok{ ::} \DataTypeTok{IO}\NormalTok{ ()}

\OtherTok{(\textgreater{}\textgreater{}) ::} \DataTypeTok{IO}\NormalTok{ () }\OtherTok{{-}\textgreater{}} \DataTypeTok{IO}\NormalTok{ () }\OtherTok{{-}\textgreater{}} \DataTypeTok{IO}\NormalTok{ ()}
\end{Highlighting}
\end{Shaded}

The type signature of \texttt{(\textgreater{}\textgreater{})} says (in simple
terms) that it's a function that takes two \texttt{IO\ ()}s and returns a shiny
new \texttt{IO\ ()}. It's a function that takes two objects and returns one.

We can apply \texttt{(\textgreater{}\textgreater{})} as an infix operator:

\begin{Shaded}
\begin{Highlighting}[]
\OtherTok{helloThenWorld ::} \DataTypeTok{IO}\NormalTok{ ()}
\NormalTok{helloThenWorld }\OtherTok{=} \FunctionTok{putStrLn} \StringTok{"hello"} \OperatorTok{\textgreater{}\textgreater{}} \FunctionTok{putStrLn} \StringTok{"world"}

\CommentTok{{-}{-} define \textasciigrave{}helloThenWorld\textasciigrave{} as putStrLn "hello" \textgreater{}\textgreater{} putStrLn "world"; and its}
\KeywordTok{type}\NormalTok{ is }\OtherTok{\textasciigrave{}IO ()\textasciigrave{}}\OperatorTok{.}
\end{Highlighting}
\end{Shaded}

That new \texttt{IO\ ()} is a data structure that represents the act of printing
``hello'', then printing ``world''.

Remember that this new one is, still, only a normal object. No printing actually
ever ``happens'' if you evaluate
\texttt{putStrLn\ "hello"\ \textgreater{}\textgreater{}\ putStrLn\ "world"}. If
you ever reach that expression in a Haskell program\ldots nothing is printed.
It's simply just taking two regular old data values, running them through a
function, and giving you a third one. The process of defining
\texttt{helloThenWorld} --- or even later evaluating it --- doesn't cause
anything to happen. They are inert data structures.

In many other languages, sequencing actions is a special part of the syntax ---
a semicolon, usually. In Haskell, sequencing is not special --- it's just a
normal function on normal data structures.

You can even make your own ``first class'' control flow!

\begin{Shaded}
\begin{Highlighting}[]
\OtherTok{when ::} \DataTypeTok{Bool} \OtherTok{{-}\textgreater{}} \DataTypeTok{IO}\NormalTok{ () }\OtherTok{{-}\textgreater{}} \DataTypeTok{IO}\NormalTok{ ()}
\NormalTok{when }\DataTypeTok{True}\NormalTok{  p }\OtherTok{=}\NormalTok{ p}
\NormalTok{when }\DataTypeTok{False}\NormalTok{ \_ }\OtherTok{=} \FunctionTok{return}\NormalTok{ ()}
\end{Highlighting}
\end{Shaded}

(\texttt{return\ ()} is an \texttt{IO\ ()} that represents the act of doing
nothing\ldots it doesn't actually have anything to do with the \texttt{return}
keyword in many other languages. It basically represents a no-op.)

\texttt{when} is just a normal function! It takes a \texttt{Bool} and an
\texttt{IO\ ()}; if the \texttt{Bool} is true, then the ``result'' is just that
same \texttt{IO\ ()}.

We can evaluate a call to
\texttt{when\ (4\ \textgreater{}\ 0)\ (putStrLn\ "it\textquotesingle{}s\ True!")}
by hand:

\begin{Shaded}
\begin{Highlighting}[]
\NormalTok{when (}\DecValTok{4} \OperatorTok{\textgreater{}} \DecValTok{0}\NormalTok{) (}\FunctionTok{putStrLn} \StringTok{"it\textquotesingle{}s True!"}\NormalTok{)}\OtherTok{ ::} \DataTypeTok{IO}\NormalTok{ ()}
\NormalTok{when }\DataTypeTok{True}\NormalTok{ (}\FunctionTok{putStrLn} \StringTok{"it\textquotesingle{}s True!"}\NormalTok{)}\OtherTok{    ::} \DataTypeTok{IO}\NormalTok{ ()  }\CommentTok{{-}{-} evaluate 4 \textgreater{} 0}
\FunctionTok{putStrLn} \StringTok{"it\textquotesingle{}s True!"}\OtherTok{                ::} \DataTypeTok{IO}\NormalTok{ ()  }\CommentTok{{-}{-} definition of when True}

\NormalTok{when (}\DecValTok{4} \OperatorTok{\textless{}} \DecValTok{0}\NormalTok{) (}\FunctionTok{putStrLn} \StringTok{"it\textquotesingle{}s True!"}\NormalTok{)}\OtherTok{ ::} \DataTypeTok{IO}\NormalTok{ ()}
\NormalTok{when }\DataTypeTok{False}\NormalTok{ (}\FunctionTok{putStrLn} \StringTok{"it\textquotesingle{}s True!"}\NormalTok{)}\OtherTok{   ::} \DataTypeTok{IO}\NormalTok{ ()  }\CommentTok{{-}{-} evaluate 4 \textless{} 0}
\FunctionTok{return}\OtherTok{ ()                            ::} \DataTypeTok{IO}\NormalTok{ ()  }\CommentTok{{-}{-} definition of when False}
\end{Highlighting}
\end{Shaded}

The above is \emph{not} an ``execution''\ldots it's an \emph{evaluation}.
Execution involves executing actions on a computer, where evaluation is simply a
reduction, like \texttt{1\ +\ 1\ ==\textgreater{}\ 2}. \texttt{when} is a
function that takes a \texttt{Bool} and an \texttt{IO\ ()} object and
\emph{evaluates} to that \texttt{IO\ ()} object when the boolean is True. But
remember, calling \texttt{when} doesn't actually execute anything! It's just a
normal function and normal expression. An \texttt{IO\ ()} goes in, and
\texttt{IO\ ()} comes out. Just a normal function on normal data. And we know
it's just a normal function, because we wrote it ourself from scratch!

You can't write \texttt{when} in this naive way in a language like, say,
Javascript:

\begin{Shaded}
\begin{Highlighting}[]
\KeywordTok{var}\NormalTok{ when }\OperatorTok{=} \KeywordTok{function}\NormalTok{(cond}\OperatorTok{,}\NormalTok{ act) \{ }\ControlFlowTok{if}\NormalTok{ (cond) \{ act}\OperatorTok{;}\NormalTok{ \} \}}\OperatorTok{;}

\FunctionTok{when}\NormalTok{(}\KeywordTok{false}\OperatorTok{,} \BuiltInTok{console}\OperatorTok{.}\FunctionTok{log}\NormalTok{(}\StringTok{"hello"}\NormalTok{))}\OperatorTok{;}
\CommentTok{// "hello" is printed, even though the condition is false}
\end{Highlighting}
\end{Shaded}

(You can simulate something that works by having \texttt{when} take functions
instead of statements\ldots but that's the point! You can't pass in a
``statement'', you have to pass in a function/data. In this sense, statements
and data behave completely differently.)

With only a basic knowledge of functional programming (using a
fold/reduce/inject, basically, or even recursion), you can easy write this
function:

\begin{Shaded}
\begin{Highlighting}[]
\FunctionTok{sequence\_}\OtherTok{ ::}\NormalTok{ [}\DataTypeTok{IO}\NormalTok{ ()] }\OtherTok{{-}\textgreater{}} \DataTypeTok{IO}\NormalTok{ ()}
\end{Highlighting}
\end{Shaded}

Which says, ``give me a list of \texttt{IO\ ()}s, and I'll give you a new
\texttt{IO\ ()} that represents executing all of those \texttt{IO\ ()}s
one-after-another''.

\begin{center}\rule{0.5\linewidth}{0.5pt}\end{center}

\textbf{Aside}

If you are curious, here is the definition of \texttt{sequence} using a fold:

\begin{Shaded}
\begin{Highlighting}[]
\FunctionTok{sequence\_}\OtherTok{ ::}\NormalTok{ [}\DataTypeTok{IO}\NormalTok{ ()] }\OtherTok{{-}\textgreater{}} \DataTypeTok{IO}\NormalTok{ ()}
\FunctionTok{sequence\_}\NormalTok{ xs }\OtherTok{=} \FunctionTok{foldr}\NormalTok{ (}\OperatorTok{\textgreater{}\textgreater{}}\NormalTok{) (}\FunctionTok{return}\NormalTok{ ()) xs}
\end{Highlighting}
\end{Shaded}

If you're familiar with folds/reduces, \texttt{return\ ()} is the ``base
value'', and \texttt{(\textgreater{}\textgreater{})} is the ``accumulating
function''.

\begin{Shaded}
\begin{Highlighting}[]
\FunctionTok{sequence\_}\NormalTok{ [}\FunctionTok{putStrLn} \StringTok{"hello"}\NormalTok{, }\FunctionTok{putStrLn} \StringTok{"world"}\NormalTok{, }\FunctionTok{putStrLn} \StringTok{"goodbye!"}\NormalTok{]}

\CommentTok{{-}{-} evaluates to:}
\FunctionTok{putStrLn} \StringTok{"hello"} \OperatorTok{\textgreater{}\textgreater{}}\NormalTok{ (}\FunctionTok{putStrLn} \StringTok{"world"} \OperatorTok{\textgreater{}\textgreater{}}\NormalTok{ (}\FunctionTok{putStrLn} \StringTok{"goodbye!"} \OperatorTok{\textgreater{}\textgreater{}} \FunctionTok{return}\NormalTok{ ()))}
\end{Highlighting}
\end{Shaded}

\begin{center}\rule{0.5\linewidth}{0.5pt}\end{center}

Note that all of these functions take anything of type \texttt{IO\ ()}\ldots so
I could really be passing in named \texttt{IO\ ()}'s, or the result of
combinators, or\ldots{}

\begin{Shaded}
\begin{Highlighting}[]
\OtherTok{hello ::} \DataTypeTok{IO}\NormalTok{ ()}
\NormalTok{hello }\OtherTok{=} \FunctionTok{putStrLn} \StringTok{"hello"}

\OtherTok{world ::} \DataTypeTok{IO}\NormalTok{ ()}
\NormalTok{world }\OtherTok{=} \FunctionTok{putStrLn} \StringTok{"world"}

\OtherTok{helloworld ::} \DataTypeTok{IO}\NormalTok{ ()}
\NormalTok{helloworld }\OtherTok{=}\NormalTok{ hello }\OperatorTok{\textgreater{}\textgreater{}}\NormalTok{ world}

\OtherTok{helloworldhelloworld ::} \DataTypeTok{IO}\NormalTok{ ()}
\NormalTok{helloworldhelloworld }\OtherTok{=} \FunctionTok{sequence\_}\NormalTok{ [hello, world, helloworld]}
\end{Highlighting}
\end{Shaded}

Remember -- nothing is being called or executed. It's just all normal functions
on normal data. The inputs are data, the outputs are data.

But wait! There are a lot of things I can do with two \texttt{IO\ ()}s besides
executing them one-after-the-other. I can\ldots merge them \emph{in parallel}!

I can write a combinator:

\begin{Shaded}
\begin{Highlighting}[]
\OtherTok{bothPar ::} \DataTypeTok{IO}\NormalTok{ () }\OtherTok{{-}\textgreater{}} \DataTypeTok{IO}\NormalTok{ () }\OtherTok{{-}\textgreater{}} \DataTypeTok{IO}\NormalTok{ ()}
\end{Highlighting}
\end{Shaded}

That takes two \texttt{IO\ ()}s and create a new shiny \texttt{IO\ ()} that
represents the act of executing them \emph{in parallel}.

Then I can also write a new \texttt{sequencePar}:

\begin{Shaded}
\begin{Highlighting}[]
\OtherTok{sequencePar ::}\NormalTok{ [}\DataTypeTok{IO}\NormalTok{ ()] }\OtherTok{{-}\textgreater{}} \DataTypeTok{IO}\NormalTok{ ()}
\end{Highlighting}
\end{Shaded}

That takes a list of \texttt{IO\ ()}s and returns a new shiny \texttt{IO\ ()}
that represents the act of executing them all \emph{in parallel}!

This is one great thing about IO-as-data: If you have a bunch of IO actions, you
have the \emph{choice} in how you want to ``sequence'' or ``combine'' them. In
Haskell, combining a bunch of actions in sequence and combining them in parallel
is just a matter of swapping out your combining function! There is \emph{no
difference} at the syntax level!

Compare this to other languages, where the syntax for sequencing statements ---
the semicolon --- and the syntax required for launching a bunch of parallel
actions is noticeably different. In Haskell, ``sequencing'' isn't a part of the
syntax (the semicolon) --- it's just \emph{a regular ol' function}!

\begin{center}\rule{0.5\linewidth}{0.5pt}\end{center}

\textbf{Aside}

\texttt{sequencePar}'s implementation is pretty much identical to
\texttt{sequence}'s, but swapping out \texttt{(\textgreater{}\textgreater{})}
for \texttt{bothPar}:

\begin{Shaded}
\begin{Highlighting}[]
\OtherTok{sequencePar ::}\NormalTok{ [}\DataTypeTok{IO}\NormalTok{ ()] }\OtherTok{{-}\textgreater{}} \DataTypeTok{IO}\NormalTok{ ()}
\NormalTok{sequencePar xs }\OtherTok{=} \FunctionTok{foldr}\NormalTok{ bothPar (}\FunctionTok{return}\NormalTok{ ()) xs}
\end{Highlighting}
\end{Shaded}

By the way, \texttt{bothPar} isn't defined by default, but we'll define it
really soon.

\begin{center}\rule{0.5\linewidth}{0.5pt}\end{center}

There are an entire wealth of combinators by which to compose and sequence and
manipulate \texttt{IO\ ()}s together. And many of them you can even write
yourself, from scratch.

There are also many ``IO action transformers'' you have access to --- one
notable one being \texttt{makePar}:\footnote{\texttt{makePar} actually exists in
  the standard Haskell libraries as \texttt{forkIO}\ldots kinda.
  \texttt{forkIO\ ::\ IO\ ()\ -\textgreater{}\ IO\ ThreadId}. Our
  \texttt{makePar} is basically \texttt{forkIO}, but with the return value
  ignored. Basically, it is
  \texttt{makePar\ x\ =\ forkIO\ x\ \textgreater{}\textgreater{}\ return\ ()}}

\begin{Shaded}
\begin{Highlighting}[]
\OtherTok{makePar ::} \DataTypeTok{IO}\NormalTok{ () }\OtherTok{{-}\textgreater{}} \DataTypeTok{IO}\NormalTok{ ()}
\end{Highlighting}
\end{Shaded}

That takes an \texttt{IO\ ()}, and ``transforms'' it into a ``parallel''
\texttt{IO\ ()}. Or rather, it takes an object representing a computer action,
and returns an object representing launching that computer action in a parallel
fork.

We can write \texttt{bothPar} ourselves, then, with this:\footnote{Note that
  this action doesn't ``wait'' for both threads to complete; all it does is
  launch the two threads.}

\begin{Shaded}
\begin{Highlighting}[]
\OtherTok{bothPar ::} \DataTypeTok{IO}\NormalTok{ () }\OtherTok{{-}\textgreater{}} \DataTypeTok{IO}\NormalTok{ () }\OtherTok{{-}\textgreater{}} \DataTypeTok{IO}\NormalTok{ ()}
\NormalTok{bothPar x y }\OtherTok{=}\NormalTok{ makePar x }\OperatorTok{\textgreater{}\textgreater{}}\NormalTok{ makePar y}
\end{Highlighting}
\end{Shaded}

Give \texttt{bothPar} two \texttt{IO\ ()}'s representing computer actions, and
it'll give you a new one that represents launching both computer actions in
parallel. To do that, simply launch them both one after the other!

Another common transformer on an \texttt{IO\ ()} is \texttt{catch}:\footnote{Again,
  the real definition here is slightly more general.}

\begin{Shaded}
\begin{Highlighting}[]
\FunctionTok{catch}\OtherTok{ ::} \DataTypeTok{IO}\NormalTok{ () }\OtherTok{{-}\textgreater{}}\NormalTok{ (}\DataTypeTok{SomeException} \OtherTok{{-}\textgreater{}} \DataTypeTok{IO}\NormalTok{ ()) }\OtherTok{{-}\textgreater{}} \DataTypeTok{IO}\NormalTok{ ()}
\CommentTok{{-}{-}       \^{}         \^{}                          \^{}}
\CommentTok{{-}{-}       |         |                          +{-}{-} modified object}
\CommentTok{{-}{-}       |         +{-}{-} handler function}
\CommentTok{{-}{-}       +{-}{-} original object}
\end{Highlighting}
\end{Shaded}

Which takes an \texttt{IO\ ()} object and a handler function, and imbues that
\texttt{IO\ ()} with ``error handling capabilities''. It returns a new
\texttt{IO\ ()} object that represents doing the same thing as the original one,
except with built-in error handling if things go wrong. Neat!

So if I used \texttt{catch\ (putStrLn\ "hello\ world")\ myHandler}\ldots I'm
``transforming'' the \texttt{IO\ ()} (\texttt{putStrLn\ "hello\ world"}),
representing printing a string to the console, into a new \texttt{IO\ ()} which
represents printing a string to the console, with built in error handling if
things go wrong for some reason.

Again --- no execution is being done. We're simply taking an object representing
an IO action, and returning a new, modified one representing a slightly
different IO action.

\begin{center}\rule{0.5\linewidth}{0.5pt}\end{center}

\textbf{Aside}

This is a pretty aside-y aside, and you can definitely skip it if you want!

One particularly important combinator I have not mentioned yet is called
``bind'': \texttt{(\textgreater{}\textgreater{}=)}

Let's say you wanted to read a line from stdin, and then print it out right
away. You can do this with \texttt{getLine\ ::\ IO\ String} and
\texttt{putStrLn\ ::\ String\ -\textgreater{}\ IO\ ()}. But wait! This doesn't
work:

\begin{Shaded}
\begin{Highlighting}[]
\FunctionTok{getLine} \OperatorTok{\textgreater{}\textgreater{}} \FunctionTok{putStrLn} \StringTok{"hello?"}
\end{Highlighting}
\end{Shaded}

\texttt{(\textgreater{}\textgreater{})} acts like a semicolon\ldots it just
sequences them together one after the other. Wouldn't it be nice if we had
something like a unix pipe? It ``sequences'' the two things, but the result of
the first can be used by the second?

Well, if \texttt{(\textgreater{}\textgreater{})} is a bash semicolon \texttt{;},
then \texttt{(\textgreater{}\textgreater{}=)} is a bash pipe
\texttt{\textbar{}}!

\begin{Shaded}
\begin{Highlighting}[]
\FunctionTok{getLine} \OperatorTok{\textgreater{}\textgreater{}=} \FunctionTok{putStrLn}
\OtherTok{                      ::} \DataTypeTok{IO}\NormalTok{ ()}
\end{Highlighting}
\end{Shaded}

does exactly what we want!

The type of \texttt{(\textgreater{}\textgreater{}=)} is:

\begin{Shaded}
\begin{Highlighting}[]
\OtherTok{(\textgreater{}\textgreater{}=) ::} \DataTypeTok{IO}\NormalTok{ a }\OtherTok{{-}\textgreater{}}\NormalTok{ (a }\OtherTok{{-}\textgreater{}} \DataTypeTok{IO}\NormalTok{ b) }\OtherTok{{-}\textgreater{}} \DataTypeTok{IO}\NormalTok{ b}
\end{Highlighting}
\end{Shaded}

And in our specific case, it is:

\begin{Shaded}
\begin{Highlighting}[]
\OtherTok{(\textgreater{}\textgreater{}=) ::} \DataTypeTok{IO} \DataTypeTok{String} \OtherTok{{-}\textgreater{}}\NormalTok{ (}\DataTypeTok{String} \OtherTok{{-}\textgreater{}} \DataTypeTok{IO}\NormalTok{ ()) }\OtherTok{{-}\textgreater{}} \DataTypeTok{IO}\NormalTok{ ()}
\end{Highlighting}
\end{Shaded}

Which says, ``give me an \texttt{IO\ String} and a function taking a
\texttt{String} and producing an \texttt{IO\ ()}, and I'll give you a shiny new
\texttt{IO\ ()}'' This might sound a little weird at first, but see how
\texttt{getLine\ ::\ IO\ String} and
\texttt{putStrLn\ ::\ String\ -\textgreater{}\ IO\ ()} fit into this, and see
how it basically works like a unix pipe in a lot of ways.

As it turns out, \texttt{(\textgreater{}\textgreater{}=)} is actually a lot more
powerful than it might seem at first. As soon as you add the
\texttt{(\textgreater{}\textgreater{}=)} combinator to your arsenal\ldots the
space of programs you can construct using various \texttt{IO\ a}'s opens up in
crazy ways. Just imagine bash with no pipes, and only semicolons! If you ever
decide to implement some system of first-class statements, and it might be
tricky to state/model imperative computations without
\texttt{(\textgreater{}\textgreater{}=)}.

\begin{center}\rule{0.5\linewidth}{0.5pt}\end{center}

\subsection{Much More}\label{much-more}

This is only a small subset of what you can do with ``statements as data''. In
fact, there are many frameworks that completely abstract over statements
entirely. For example, you can ``construct'' a system declaratively using a
simple DSL, and never even worry about statements or IO. You can specify an
\emph{entire program}, with a full description of its interactions, without ever
even \emph{touching} the IO type. The DSL might provide you with a way to
specify a high-level overlook of your program in simple terms. The DSL might be
abstracting/wrapping over \emph{complex} IO actions, and all you ever see is the
simple API.

And then, you might have a function: \texttt{DSL\ -\textgreater{}\ IO\ ()}.
Construct the elaborate high-level thing in simple terms\ldots and then, at the
end, \emph{convert it to an \texttt{IO\ ()} object}. That you can copy, or
clone, or throw into a function, or do \emph{anything} we just mentioned here!

If that DSL is your entire program, then you also have certain guarantees about
what IO your program can even do, if the DSL library forbids users from mixing
in arbitrary IO.

\section{Execution}\label{execution}

That's nice and all. We see how having first class statements as data is useful
for manipulation and abstractions and stuff. But it all seems kind of useless if
our data structures remain inert and don't actually do anything.

Luckily, one can think of a Haskell compiler as a giant function:
\texttt{IO\ ()\ -\textgreater{}\ Binary}. Give the Haskell compiler an
\texttt{IO\ ()}, and it'll convert it to a ``binary'' for a given
architecture/computer/CPU. It ``translates'' the \emph{representation of a
computation} into concrete bytecode that a computer can actually execute.

Your computer can then execute that generated binary, and\ldots off we go!

Every Haskell program by convention compiles \emph{one single \texttt{IO\ ()}}.
That is, you might have a \emph{bunch} of \texttt{IO\ a}s in your program, but
you ``offer the compiler'' \emph{one} \texttt{IO\ ()} for it to compile, and it
compiles it for you. So if you have a lot of different IO computations you wish
to do, you basically continually sequence, merge, pipe, combine, transform, etc.
them until you get one final \texttt{IO\ ()} which represents your entire
desired computation. And then you name it ``main''. When your compiler compiles
the Haskell file, it'll find the \texttt{IO\ ()} named ``main'', and compile
that one.

In a way, one can think of Haskell as a very elaborate metaprogramming system,
providing a DSL to ``generate'' byte code.

By the way, did you see how \texttt{IO\ ()} is \emph{completely separate} from
its binary representation? Hm. Why do we have to compile it to binary, anyway?

Enter in \emph{other Haskell compilers}, like
\href{https://github.com/ghcjs/ghcjs}{ghcjs} --- instead of being
\texttt{IO\ ()\ -\textgreater{}\ Binary}, it's
\texttt{IO\ ()\ -\textgreater{}\ Javascript}! That is, give it \emph{any} ol'
\texttt{IO\ ()} (the same one that you would design for a CPU/processor), and
instead of translating it into bytecode/binary, it translates it into
Javascript! This is another power of having \texttt{IO\ ()} being its own,
abstract object that is independent of the architecture or computer that it will
be eventually run on. Because it's so abstract\ldots it can be compiled and run
really on \emph{anything}!

\section{Haskell}\label{haskell}

If you have any questions or comments, feel free to leave a comment, drop by
freenode's \emph{\#haskell}, \emph{\#nothaskell}, or \emph{\#haskell-beginners},
or find me on \href{https://twitter.com/mstk}{twitter}.

In this post I've suggested that \texttt{IO\ ()} is some sort of data structure
that stores the ``action'' it represents in some abstract way. If you're curious
on what this representation/storage might look like in concrete terms,
\href{http://chris-taylor.github.io/blog/2013/02/09/io-is-not-a-side-effect/}{Chris
Taylor has a post} on what you might see if you ``peek into'' the internal
representation of (a possible implementation of) an IO action type --- you could
even use this to implement first-class statements in your language of choice!
\footnote{For performance reasons, the way IO is implemented in that article
  actually isn't the way it's implemented in the popular Haskell compiler
  \emph{GHC}. GHC's implementation is best described as ``hacky'', and doesn't
  really line up too well with the semantic picture of what \texttt{IO\ ()} is
  supposed to represent. But remember that this is really just an (admittedly
  ugly) \emph{implementation detail}. The outward-facing API that it offers for
  the \texttt{IO\ ()} type works as you would expect, of course. One would hope,
  at least! In any case, it's important to remember the difference between
  \emph{Haskell the language}, and what the IO type is supposed to
  ``represent'', and \emph{Haskell the implementation}, which provides the
  semantics of the language, abstracting over ugly implementation details.}

This post is a distillation of concepts I have mentioned in
\href{http://blog.jle.im/entry/the-compromiseless-reconciliation-of-i-o-and-purity}{some
other}
\href{http://blog.jle.im/entry/inside-my-world-ode-to-functor-and-monad}{blog
posts} in the past; I've had a lot of new thoughts after writing both of them
and I figured I'd condense them and make a new post summarizing the new ideas in
a new and more concise way, to have them all in one neat place. Anyways, if you
want to go into this topic in more detail, those posts above might help!

If you're interested in learning Haskell, try picking up
\href{http://www.learnyouahaskell.com/}{Learn You a Haskell} and giving it a
read, it's pretty accessible!
\href{https://github.com/bitemyapp/learnhaskell}{bitemyapp's guide} also lays
out a nice roadmap for learning Haskell.

Also, I encourage you to try to implement your own system of ``first class IO''
in languages in which it is possible! Like a normal data structure (like in the
Chris Taylor post), or an abstracted function call. I'd love to hear of any
results or attempts you've made implementing this in your language of choice
(even if Haskell is your language of choice, you can write a \texttt{MyIO} type
:D ); let me know in the comments or via
\href{https://twitter.com/mstk}{twitter}!

(Credit to \href{https://twitter.com/computionist}{computionist} and
\href{https://twitter.com/bitemyapp}{bitemyapp} for proofreading/helpful
suggestions)

\section{Signoff}\label{signoff}

Hi, thanks for reading! You can reach me via email at
\href{mailto:justin@jle.im}{\nolinkurl{justin@jle.im}}, or at twitter at
\href{https://twitter.com/mstk}{@mstk}! This post and all others are published
under the \href{https://creativecommons.org/licenses/by-nc-nd/3.0/}{CC-BY-NC-ND
3.0} license. Corrections and edits via pull request are welcome and encouraged
at \href{https://github.com/mstksg/inCode}{the source repository}.

If you feel inclined, or this post was particularly helpful for you, why not
consider \href{https://www.patreon.com/justinle/overview}{supporting me on
Patreon}, or a \href{bitcoin:3D7rmAYgbDnp4gp4rf22THsGt74fNucPDU}{BTC donation}?
:)

\end{document}
