\documentclass[]{article}
\usepackage{lmodern}
\usepackage{amssymb,amsmath}
\usepackage{ifxetex,ifluatex}
\usepackage{fixltx2e} % provides \textsubscript
\ifnum 0\ifxetex 1\fi\ifluatex 1\fi=0 % if pdftex
  \usepackage[T1]{fontenc}
  \usepackage[utf8]{inputenc}
\else % if luatex or xelatex
  \ifxetex
    \usepackage{mathspec}
    \usepackage{xltxtra,xunicode}
  \else
    \usepackage{fontspec}
  \fi
  \defaultfontfeatures{Mapping=tex-text,Scale=MatchLowercase}
  \newcommand{\euro}{€}
\fi
% use upquote if available, for straight quotes in verbatim environments
\IfFileExists{upquote.sty}{\usepackage{upquote}}{}
% use microtype if available
\IfFileExists{microtype.sty}{\usepackage{microtype}}{}
\usepackage[margin=1in]{geometry}
\ifxetex
  \usepackage[setpagesize=false, % page size defined by xetex
              unicode=false, % unicode breaks when used with xetex
              xetex]{hyperref}
\else
  \usepackage[unicode=true]{hyperref}
\fi
\hypersetup{breaklinks=true,
            bookmarks=true,
            pdfauthor={},
            pdftitle={Introducing ``in Code''!},
            colorlinks=true,
            citecolor=blue,
            urlcolor=blue,
            linkcolor=magenta,
            pdfborder={0 0 0}}
\urlstyle{same}  % don't use monospace font for urls
% Make links footnotes instead of hotlinks:
\renewcommand{\href}[2]{#2\footnote{\url{#1}}}
\setlength{\parindent}{0pt}
\setlength{\parskip}{6pt plus 2pt minus 1pt}
\setlength{\emergencystretch}{3em}  % prevent overfull lines
\setcounter{secnumdepth}{0}

\title{Introducing ``in Code''!}

\begin{document}
\maketitle

\% Justin Le \% September 17, 2013

\emph{Originally posted on
\textbf{\href{https://blog.jle.im/entry/introducing-in-code.html}{in Code}}.}

Throughout my time programming and developing, I've noticed that there are few
legitimately useful sources of practical and beneficial knowledge on the subject
on the internet. There are good books. There is
\href{http://www.stackoverflow.com}{Stack Overflow}. And there are the blogs.

I've probably learned more useful information from blogs than I have from nearly
any other source --- from small things like how to fix that compile error on
ffmpeg or teaching me my first steps in learning tools I now use every day like
git. From blogs I've learned not just small things that help me here and there,
but also the building blocks that I have to thank for allowing me to learn so
much of what I know now.

In addition to these, blogs are the theatre of many gifted essayists, who are
always enthusiastic to share their insight from the deeper aspects of what
programming actually means, and their roles in the construction of the world we
live in today.

Having worked on my fair share of projects, I've accumulated some meager, humble
practical knowledge from the field over these short years. So this is my
opportunity to finally give back. My hope is first to fill in all of those small
holes in knowledge that happen to slip through and can sometimes only be found
with the perfect google search. Second to share a bit of my discoveries along my
journeys in development and any relevant insights from my studies that may be
useful to others. Third, to share any mature open source projects I'm working on
that may be of use to the world. And forth, to maybe shine some light on greater
themes that I have noticed when looking back on things.

It's also been said that blogs are a good way for someone to help themselves
organize their thoughts and kind of put them all together. Maybe if I log my
progress on projects as I go along, I can look back and refer to them later. And
maybe --- just maybe --- they can be useful to other humans.

So anyways, this is it. Welcome to \textbf{in Code}!

\section{Signoff}\label{signoff}

Hi, thanks for reading! You can reach me via email at
\href{mailto:justin@jle.im}{\nolinkurl{justin@jle.im}}, or at twitter at
\href{https://twitter.com/mstk}{@mstk}! This post and all others are published
under the \href{https://creativecommons.org/licenses/by-nc-nd/3.0/}{CC-BY-NC-ND
3.0} license. Corrections and edits via pull request are welcome and encouraged
at \href{https://github.com/mstksg/inCode}{the source repository}.

If you feel inclined, or this post was particularly helpful for you, why not
consider \href{https://www.patreon.com/justinle/overview}{supporting me on
Patreon}, or a \href{bitcoin:3D7rmAYgbDnp4gp4rf22THsGt74fNucPDU}{BTC donation}?
:)

\end{document}
