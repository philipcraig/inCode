\documentclass[]{article}
\usepackage{lmodern}
\usepackage{amssymb,amsmath}
\usepackage{ifxetex,ifluatex}
\usepackage{fixltx2e} % provides \textsubscript
\ifnum 0\ifxetex 1\fi\ifluatex 1\fi=0 % if pdftex
  \usepackage[T1]{fontenc}
  \usepackage[utf8]{inputenc}
\else % if luatex or xelatex
  \ifxetex
    \usepackage{mathspec}
    \usepackage{xltxtra,xunicode}
  \else
    \usepackage{fontspec}
  \fi
  \defaultfontfeatures{Mapping=tex-text,Scale=MatchLowercase}
  \newcommand{\euro}{€}
\fi
% use upquote if available, for straight quotes in verbatim environments
\IfFileExists{upquote.sty}{\usepackage{upquote}}{}
% use microtype if available
\IfFileExists{microtype.sty}{\usepackage{microtype}}{}
\usepackage[margin=1in]{geometry}
\ifxetex
  \usepackage[setpagesize=false, % page size defined by xetex
              unicode=false, % unicode breaks when used with xetex
              xetex]{hyperref}
\else
  \usepackage[unicode=true]{hyperref}
\fi
\hypersetup{breaklinks=true,
            bookmarks=true,
            pdfauthor={},
            pdftitle={Advent of Code 2017! Ongoing solutions and explanations},
            colorlinks=true,
            citecolor=blue,
            urlcolor=blue,
            linkcolor=magenta,
            pdfborder={0 0 0}}
\urlstyle{same}  % don't use monospace font for urls
% Make links footnotes instead of hotlinks:
\renewcommand{\href}[2]{#2\footnote{\url{#1}}}
\setlength{\parindent}{0pt}
\setlength{\parskip}{6pt plus 2pt minus 1pt}
\setlength{\emergencystretch}{3em}  % prevent overfull lines
\setcounter{secnumdepth}{0}

\title{Advent of Code 2017! Ongoing solutions and explanations}

\begin{document}
\maketitle

\% Justin Le \% December 7, 2017

\emph{Originally posted on
\textbf{\href{https://blog.jle.im/entry/advent-of-code-2017.html}{in Code}}.}

Just a short post to share that I started a github repository of my
\href{https://github.com/mstksg/advent-of-code-2017}{Advent of Code 2017
Solutions}, as I write them!

I also am including my
\href{https://github.com/mstksg/advent-of-code-2017/blob/master/reflections.md}{reflections}
and explanations on my solutions, explaining my thought processes and how the
solutions work.

Yes I definitely spent a bit too much time writing the executable, which is an
automated (cached) downloader, test suite runner (on sample inputs), and
benchmark suite.

I originally was only going to casually try the problems (like I did last year),
but I hit a decent global rank by accident on Day 4 (which was very suited for
Haskell!), and since then I've been taking things seriously to try to aim for
the global leaderboard (top 100). This is a struggle for me because I'm not
really the \emph{fastest} algorithm person, but I think it's a fun goal for me
to try to hit this year.

Wish me luck! And if you haven't started yet, it's not too late to join in the
fun! \href{https://twitter.com/glguy}{glguy} has been maintaining the
semi-official \href{adventofcode.com/2017/leaderboard/private}{Haskell
Leaderboard} (join code \texttt{43100-84040706}) -- come join us!

\section{Signoff}\label{signoff}

Hi, thanks for reading! You can reach me via email at
\href{mailto:justin@jle.im}{\nolinkurl{justin@jle.im}}, or at twitter at
\href{https://twitter.com/mstk}{@mstk}! This post and all others are published
under the \href{https://creativecommons.org/licenses/by-nc-nd/3.0/}{CC-BY-NC-ND
3.0} license. Corrections and edits via pull request are welcome and encouraged
at \href{https://github.com/mstksg/inCode}{the source repository}.

If you feel inclined, or this post was particularly helpful for you, why not
consider \href{https://www.patreon.com/justinle/overview}{supporting me on
Patreon}, or a \href{bitcoin:3D7rmAYgbDnp4gp4rf22THsGt74fNucPDU}{BTC donation}?
:)

\end{document}
